
\documentclass[a4paper,12pt,oneside]{article}

\usepackage{amsmath}
\usepackage[utf8x]{inputenc}
\usepackage[T1]{fontenc}
\usepackage{setspace}
\usepackage[left=3.5cm,top=3cm,right=2.5cm,bottom=2.5cm]{geometry}
\usepackage{subfigure}
\usepackage{graphicx}

\setlength{\parskip}{8pt plus 8pt minus 8pt}
\setlength{\itemsep}{12pt}

\usepackage{color,hyperref,colortbl}
\definecolor{darkblue}{rgb}{0.0,0.0,0.3}
\definecolor{gray}{rgb}{0.8,0.8,0.8}
\hypersetup{colorlinks,
            linkcolor=darkblue,urlcolor=darkblue,
            anchorcolor=darkblue,citecolor=darkblue}

\bibliographystyle{plain}

\title{Implementation of Collocation Extraction in Unitex}
\author{B. Burak Arslan}
\date{\today}

\begin{document}
\maketitle
\section{Introduction}
Collocations (aka compound words and also multi-word expressions), as put
in\cite{smadja93} are ``recurrent combinations of words that co-occur more often
than expected by chance and that correspond to arbitrary word usages''. While
this purely statistical definition is correct, it overlooks most of the other
properties of collocations thus needs further refinement depending on the
application.

According to \cite{SeretanWehrli2006ACL}, collocations should rather be defined
as: ``co-occurrence of two or more lexical items as realizations of structural
elements within a given syntactic pattern'' or ``a sequence of two or more
consecutive words, that has characteristics of a syntactic and semantic unit''.

Study of the collocations in the pre-computing era showed that collocations have
particular statistical distributions. Most present challenges to second language
speakers, as most of the collocations can't be translated directly to a target
language\footnote{e.g. think about the direct translation of the collocation
``dead serious'' to any other language you know}

Collocations are language dependent and can only be learned by observing their
occurrence in language use; they are otherwise not
predictable\cite{seretan2003}. Part of what is termed as \emph{Multi Word
Expressions}, uses of collocation extraction systems are exposed in detail in
\cite{sag02multiword}.

One of the main challenges in collocation extraction is the fact that the
collocations vary wildly in form. This suggests not only that collocations may
appear under very different forms, but also one single collocation may have many
forms depending on its context. For example, the collocation ``to make a
decision'' may appear as ``\textit{the decisions} he \textit{made}'', or ``he
\textit{made an} important \textit{decision}'', or ``he is about \textit{to make
a decision}''.

That's why a series of operations that exploit different characteristics of
collocations must be applied to the corpora in order to obtain most accurate
results.

\section{Extraction of collocations in Unitex}
We can say that, roughly, a hybrid collocation extraction system operates in two
steps: Firstly, collocation candidates are extracted. Those are then fed to a
system that apply various statistical tests to the distributions of the given
collocations. The initial results may be combined or eliminated. This results in
a list of multi-word collocations.

\subsection{Extracting collocation candidates}
There are two possible approaches to this problem: 

\begin{description}
\item[Deep parsing:] Apply syntactic analysis to the text (using a deep parser
like Fips\cite{fips2007}) and extract collocation candidates basing on syntactic
relations between words \cite{SeretanWehrli2006ACL}. This is a relatively
complicated approach that requires a full-blown grammar and its parser for the
target language.

\item[Shallow Parsing:] Treat the text as a linear string of words, and for a
given word, count all the occurences of other words in the vicinity of the given
word. 

But, before making relatively complex operations on raw text, it is always a
good idea to apply a few normalizations to the text. So, operations like marking
sentence boundaries, lemmatization, POS-tagging etc. simplifies greatly the
lives of processess that may follow\cite{smadja93}. These operations are called
\emph{shallow parsing} operations, because they don't analyze the structure of
the whole sentence.

While this approach is simpler to adopt, because of combinatorial explosion it
is likely to cause, it is desirable to restrain the definition of the
``neighborhood of a word'' to a small number. Smadja, in \cite{smadja93} states
that 5 is a reasonable limit while Seretan and Wehrli argue, in
\cite{SeretanWehrli2006ACL}, that this limit is too low. 

Buth both seemingly agree that all words that make up a collocation reside in
the same sentence.

\end{description}

For Unitex, we've chosen to implement a candidate extraction algorith that will
work on the text automata, which contains all interpretations of a surface string.
This will result in a more comprehensive set of collocation candidates.

\subsection{Statistical Tests}

\bibliography{rap.bib}

\end{document}



